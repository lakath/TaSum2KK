P� vores indledende projektetableringsm�de blev vi enige om at vi alle var klar over projektets omfang, men m�tte ogs� indse at vi med al sandsynlighed ikke n�r at levere et f�rdigt og brugbart produkt indenfor den afsatte tid.
Derfor har vi valgt at vi gerne vil arbejde med en agil metode, der bl.a. sikrer at vi n�r at l�re noget om udviklingsprocessen involveret i det, og fordi det ogs� har v�re salgstalen overfor Kollegiekontoret, s� de f�r en smag for den side af udviklingsverdenen inden de endeligt f�r l�st deres behov for en hjemmeside.

%bla bla bla scrum - mere begrundelse, interessenter

\subsection{Turboanalyse}
\begin{tabular}{| l | l | l | l |}
\hline
M�l og vilk�r & S�rlig styrke & S�rlig svaghed & Mulige beslutninger \\ \hline
Teknikken &  & Usikkerhed om nuv�rende anvendt teknik & Vi kan starte fra bunden, fremfor at bygge ovenp� eksisterende \\ \hline
Udviklerne &  & Uerfarne & Vi skal v�re forberedte p� at f� hj�lp fra undervisere. \\ \hline
Resultat & Skal ikke blive f�rdigt, da det ikke er m�let &  & Fokuser p� at l�re noget \\ \hline
Brugerne & Brugerne har sv�rt ved at frav�lge produktet &  &  \\ \hline
Omgivelserne &  & Projektgruppen har andre opgaver under forl�bet & Afstem forventninger og prioriteter \\ \hline
& \multicolumn{3}{| c |}{Projektets strategi} \\ \hline
Sammenfatning & \multicolumn{3}{| l |}{ Vi starter umiddelbart udviklingsprojektet fra bar bund, og satser ikke p� at levere et f�rdig produkt, men at vi l�rer noget af proccessen ved et udviklingsprojekt. Vi vil ogs� v�re eksperimenterende i id�erne s� der kan komme noget brugbart og inspirerende ud af projektet for alle parter. } \\ \hline
\end{tabular}
