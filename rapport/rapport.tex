\documentclass[12pt, a4paper]{report}
\title{SUM Projekt - Kollegiekontoret}
\author{Silas Fontain Jakobsen \and Tina Jensen \and Andreas Kristoffersen \and Jannie Gier Larsen \and Erik Holm Sejersen}
\date{21. Maj 2012}

% latex crash course: http://www.haptonstahl.org/latex/

\begin{document}
\begin{titlepage}
\maketitle
\end{titlepage}

\begin{abstract}
Forord
\end{abstract}

\tableofcontents

\chapter*{Forundersøgelse}
% SE PRÆSENTATION OM FORUNDERSØGELSE 3

\section*{Forberedelse}
 \begin{itemize}
   \item Udgangspunkt
   \item Organisering
   \item Interessentanalyse
   \item Metode
 \end{itemize}

\section*{Projektgrundlag}
\subsection{Udgangspunkt}
\subsubsection{Baggrund}
Vi har i forbindelse med en stillet eksamensopgave taget kontakt til kollegiekontoret
\subsubsection{Opgaven og formål}
\subsubsection{Økonomiske og tekniske rammer}
\subsubsection{Kritiske faktorer}

\subsection{Organisering}
\subsubsection{Projektets organisering}
\subsubsection{Ressourcer}
\subsubsection{Interessenter}
\subsubsection{Aftaler og koordinering}

\subsection{Metode}
\subsubsection{Overodnet fremgangsmåde}
\subsubsection{Plan}
\subsubsection{Teknikker og beskrivelsesværktøjer}
\subsubsection{Arbejdsform}

\subsection{Underskrifter}

\section*{Fokusering}
 \begin{itemize}
   \item Virksomhedsbeskrivelse

\begin{itemize}

   \item Interne forhold

   \item Omgivelser



   \item Foretningsstrategi og IT-strategi

   \item SWOT-analyse

\end{itemize}
   \item Projektets arbejdsområder
 \end{itemize}

\section*{Fordybelse}
\begin{itemize}
\item Dataindsamling, beskrivelser af nuværende arbejdspraksis
\begin{itemize}
\item Referat af observationer og interviews
\end{itemize}
\item Rige billeder
\item Diagnostiske kort, mål, problemer, behov og idéer til løsning
\end{itemize}

\section*{Fornyelse}
\begin{itemize}
\item Visioner om den samlede forandring
\Strategi og plan for realisering
\item Eksperimenter med prototyper
\end{itemize}

\section*{Konklusion på forundersøgelse}
Valg af vision

\chapter*{Processen}

\section*{Projektetablering}
På vores indledende projektetableringsmøde blev vi enige om at vi alle var klar over projektets omfang, men måtte også indse at vi med al sandsynlighed ikke når at levere et færdigt og brugbart produkt indenfor den afsatte tid.
Derfor har vi valgt at vi gerne vil arbejde med en agil metode, der bl.a. sikrer at vi når at lære noget om udviklingsprocessen involveret i det, og fordi det også har være salgstalen overfor Kollegiekontoret, så de får en smag for den side af udviklingsverdenen inden de endeligt får løst deres behov for en hjemmeside.

%bla bla bla scrum - mere begrundelse, interessenter

\subsection*{Turboanalyse}
\begin{tabular}{| l | l | l | l |}
\hline
Mål og vilkår & Særlig styrke & Særlig svaghed & Mulige beslutninger \\ \hline
Teknikken & & & \\ \hline
Udviklerne & & & \\ \hline
Resultat & & & \\ \hline
Omgivelserne & & & \\ \hline
& \multicolumn{3}{| c |}{Projektets strategi} \\ \hline
Sammenfatning & \\ \hline
\end{tabular}

\section*{Projektplan}

\section*{Iterationer}

\section*{Kvalitetsbeskrivelse}

\section*{Projektevaluering}

\chapter*{Produktet}

\section*{Kravspecifikation}

\section*{Testcases}

\section*{Modeller}

\section*{Design}

\section*{Kode}

\section*{Foranalyse / Projektgrundlag}



\end{document}

