\documentclass[12pt, a4paper]{report}
\setcounter{secnumdepth}{-1}
\title{SUM Projekt - Kollegiekontoret}
\author{Silas Fontain Jakobsen \and Tina Jensen \and Andreas Kristoffersen \and Jannie Gier Larsen \and Erik Holm Sejersen}
\date{21. Maj 2012}

% latex crash course: http://www.haptonstahl.org/latex/

\begin{document}
\begin{titlepage}
\maketitle
\end{titlepage}

\begin{abstract}
Forord
\end{abstract}

\tableofcontents

\chapter{Forunders�gelse}
% SE PR�SENTATION OM FORUNDERS�GELSE 3

\section{Forberedelse}
 \begin{itemize}
   \item Udgangspunkt
   \item Organisering
   \item Interessentanalyse
   \item Metode
 \end{itemize}

\section{Projektgrundlag}
\subsection{Udgangspunkt}
\subsubsection{Baggrund}
Vi har i forbindelse med en stillet eksamensopgave taget kontakt til Kollegiekontoret, da de har et kommende IT projekt med henblik p� at f� opgraderet deres hjemmeside til udlejning af ungdomsboliger.
\subsubsection{Opgaven og form�l}
Vores opgaver best�r i at afd�kke Kollegiekontorets behov til en ny hjemmeside, der er tidssvarende og giver potentielle lejere et bedre overblik over deres muligheder for at leje en ungdomsbolig.
\subsubsection{�konomiske og tekniske rammer}
\subsubsection{Kritiske faktorer}

\subsection{Organisering}
\subsubsection{Projektets organisering}
\subsubsection{Ressourcer}
\subsubsection{Interessenter}
Vi vurderer, at der p� projektet findes f�lgende relevante og klart definerede interessenter:
\begin{itemize}
\item Kunde: Kollegiekontoret
\item Leverand�r: Team Awesome
\item Projektleder: Tina Jensen
\item Udviklere, testere: Team Awesome
\item
\subsubsection{Aftaler og koordinering}

\subsection{Metode}
\subsubsection{Overodnet fremgangsm�de}
\subsubsection{Plan}
\subsubsection{Teknikker og beskrivelsesv�rkt�jer}
\subsubsection{Arbejdsform}

\subsection{Underskrifter}

\section{Fokusering}
 \begin{itemize}
   \item Virksomhedsbeskrivelse

\begin{itemize}

   \item Interne forhold

   \item Omgivelser

   \item Foretningsstrategi og IT-strategi

   \item SWOT-analyse

\end{itemize}
   \item Projektets arbejdsomr�der
 \end{itemize}

\section{Fordybelse}
\begin{itemize}
\item Dataindsamling, beskrivelser af nuv�rende arbejdspraksis
\begin{itemize}
\item Referat af observationer og interviews
\end{itemize}
\item Rige billeder
\item Diagnostiske kort, m�l, problemer, behov og id�er til l�sning
\end{itemize}

\section{Fornyelse}
\begin{itemize}
\item Visioner om den samlede forandring
\Strategi og plan for realisering
\item Eksperimenter med prototyper
\end{itemize}

\section{Konklusion p� forunders�gelse}
Valg af vision

\chapter{Processen}

\section{Projektetablering}
P� vores indledende projektetableringsm�de blev vi enige om at vi alle var klar over projektets omfang, men m�tte ogs� indse at vi med al sandsynlighed ikke n�r at levere et f�rdigt og brugbart produkt indenfor den afsatte tid.
Derfor har vi valgt at vi gerne vil arbejde med en agil metode, der bl.a. sikrer at vi n�r at l�re noget om udviklingsprocessen involveret i det, og fordi det ogs� har v�re salgstalen overfor Kollegiekontoret, s� de f�r en smag for den side af udviklingsverdenen inden de endeligt f�r l�st deres behov for en hjemmeside.

%bla bla bla scrum - mere begrundelse, interessenter

\subsection{Turboanalyse}
\begin{tabular}{| l | l | l | l |}
\hline
M�l og vilk�r & S�rlig styrke & S�rlig svaghed & Mulige beslutninger \\ \hline
Teknikken & & & \\ \hline
Udviklerne & & & \\ \hline
Resultat & & & \\ \hline
Omgivelserne & & & \\ \hline
& \multicolumn{3}{| c |}{Projektets strategi} \\ \hline
Sammenfatning & \\ \hline
\end{tabular}

\section{Projektplan}

\section{Iterationer}

\section{Kvalitetsbeskrivelse}

\section{Projektevaluering}

\chapter{Produktet}

\section{Kravspecifikation}

\section{Testcases}

\section{Modeller}

\section{Design}

\section{Kode}

\section{Foranalyse / Projektgrundlag}



\end{document}

